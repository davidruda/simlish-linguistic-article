% This must be in the first 5 lines to tell arXiv to use pdfLaTeX, which is strongly recommended.
\pdfoutput=1
% In particular, the hyperref package requires pdfLaTeX in order to break URLs across lines.

\documentclass[11pt]{article}

% Change "review" to "final" to generate the final (sometimes called camera-ready) version.
% Change to "preprint" to generate a non-anonymous version with page numbers.
\usepackage[preprint]{acl}

% Standard package includes
\usepackage{times}
\usepackage{latexsym}

% For proper rendering and hyphenation of words containing Latin characters (including in bib files)
\usepackage[T1]{fontenc}
% For Vietnamese characters
% \usepackage[T5]{fontenc}
% See https://www.latex-project.org/help/documentation/encguide.pdf for other character sets

% This assumes your files are encoded as UTF8
\usepackage[utf8]{inputenc}

% This is not strictly necessary, and may be commented out,
% but it will improve the layout of the manuscript,
% and will typically save some space.
\usepackage{microtype}

% This is also not strictly necessary, and may be commented out.
% However, it will improve the aesthetics of text in
% the typewriter font.
\usepackage{inconsolata}

%Including images in your LaTeX document requires adding
%additional package(s)
\usepackage{graphicx}

% I ADDED THIS
% Display IPA symbols
\usepackage{tipa}

% If the title and author information does not fit in the area allocated, uncomment the following
%
%\setlength\titlebox{<dim>}
%
% and set <dim> to something 5cm or larger.

\title{Simlish: The language of The~Sims}

% Author information can be set in various styles:
% For several authors from the same institution:
% \author{Author 1 \and ... \and Author n \\
%         Address line \\ ... \\ Address line}
% if the names do not fit well on one line use
%         Author 1 \\ {\bf Author 2} \\ ... \\ {\bf Author n} \\
% For authors from different institutions:
% \author{Author 1 \\ Address line \\  ... \\ Address line
%         \And  ... \And
%         Author n \\ Address line \\ ... \\ Address line}
% To start a separate ``row'' of authors use \AND, as in
% \author{Author 1 \\ Address line \\  ... \\ Address line
%         \AND
%         Author 2 \\ Address line \\ ... \\ Address line \And
%         Author 3 \\ Address line \\ ... \\ Address line}

\author{David Ruda \\
  Charles University \\
%  Affiliation / Address line 2 \\
%  Affiliation / Address line 3 \\
  \texttt{rudad@matfyz.cz}
%  \\\And
%  Second Author \\
%  Affiliation / Address line 1 \\
%  Affiliation / Address line 2 \\
%  Affiliation / Address line 3 \\
%  \texttt{email@domain} \\
}

%\author{
%  \textbf{First Author\textsuperscript{1}},
%  \textbf{Second Author\textsuperscript{1,2}},
%  \textbf{Third T. Author\textsuperscript{1}},
%  \textbf{Fourth Author\textsuperscript{1}},
%\\
%  \textbf{Fifth Author\textsuperscript{1,2}},
%  \textbf{Sixth Author\textsuperscript{1}},
%  \textbf{Seventh Author\textsuperscript{1}},
%  \textbf{Eighth Author \textsuperscript{1,2,3,4}},
%\\
%  \textbf{Ninth Author\textsuperscript{1}},
%  \textbf{Tenth Author\textsuperscript{1}},
%  \textbf{Eleventh E. Author\textsuperscript{1,2,3,4,5}},
%  \textbf{Twelfth Author\textsuperscript{1}},
%\\
%  \textbf{Thirteenth Author\textsuperscript{3}},
%  \textbf{Fourteenth F. Author\textsuperscript{2,4}},
%  \textbf{Fifteenth Author\textsuperscript{1}},
%  \textbf{Sixteenth Author\textsuperscript{1}},
%\\
%  \textbf{Seventeenth S. Author\textsuperscript{4,5}},
%  \textbf{Eighteenth Author\textsuperscript{3,4}},
%  \textbf{Nineteenth N. Author\textsuperscript{2,5}},
%  \textbf{Twentieth Author\textsuperscript{1}}
%\\
%\\
%  \textsuperscript{1}Affiliation 1,
%  \textsuperscript{2}Affiliation 2,
%  \textsuperscript{3}Affiliation 3,
%  \textsuperscript{4}Affiliation 4,
%  \textsuperscript{5}Affiliation 5
%\\
%  \small{
%    \textbf{Correspondence:} \href{mailto:email@domain}{email@domain}
%  }
%}

\begin{document}
\maketitle
\begin{abstract}
%This document is a supplement to the general instructions for *ACL authors. It contains instructions for using the \LaTeX{} style files for ACL conferences.
%The document itself conforms to its own specifications, and is therefore an example of what your manuscript should look like.
%These instructions should be used both for papers submitted for review and for final versions of accepted papers.
Perhaps the most famous fictional language in video games, Simlish\footnote{\url{https://en.wikipedia.org/wiki/Simlish}} is an essential part of the success of The~Sims and certainly a huge reason why so many players fell in love with the games.
In this article, we will explore its origins, try to examine the language from a linguistic point of view, discuss some of the interesting aspects it has and compare it to other fictional languages.
\end{abstract}

\section*{Introduction to The~Sims}

Simlish is an artificial language created for the video game series The~Sims.
For those who have never heard of it,
The~Sims is a series of life simulation games where players control virtual people called ``sims'' and live their lives ``just like in the real world''. The gameplay is very open-ended as there is no specific goal to achieve. Players can choose what they want to do with their sims --- have big families, build relationships, pursue careers, fulfill sims's desires, build houses, create stories and way more --- the possibilities are endless.

After the great success of the first game released in 2000, three more sequels were released, dozens of expansion packs, and a number of spin-offs.
The~Sims~4 is the latest installment in the series, and it is still being updated with new content and expansions up to this date.
Overall, with around 200 million copies sold worldwide, there is no doubt that The~Sims is one of the most successful video game series of all time.

\section*{How Simlish was created}

Will~Wright, a game designer and the creator of The~Sims\footnote{``The~Sims'' is the name of both the first game and the entire series.}, wanted to avoid using a real language because he thought that the small number of voice lines sims would say would become repetitive very quickly. Also, there would be a need for translations into other languages, which would add additional costs. Instead, he wanted to have a language that anyone could ``understand'' and resonate with sims's emotions, regardless of their native language. Something that would fit the distinctive style of the game while still leaving the exact meaning of what was said open to the player's imagination.

Initially, Wright and his team considered using musical instruments, but this idea was quickly scrapped as it did not really convey the human side of the sims. Then, they experimented with mixing real languages like Ukrainian, Navajo, Tagalog and Estonian into one mishmash kind of language. But this also did not work well because the voice actors had problems making it sound fluent and natural \cite{adams2011elvish, barnes2020sims}.

After numerous unsuccessful attempts and some frustration, one of the voice actors suggested trying an improvisation game called ``foreign poet'', where the actor tells a poem in impassioned gibberish and the listener has to interpret it in English. Surprisingly, this gibberish seemed to be exactly what Wright was looking for and that is how the first bits of Simlish were born\footnote{The very first bits of Simlish were created for Wright's game SimCopter (1996), but its use skyrocketed into fame with The~Sims \cite{bracchi2023invented}.} \cite{kilbane2020history}.

\section*{Phonetics and Phonology}

Phonetically, Simlish sounds a lot like American~English. It contains some recognizable American~English sounds like the r-colored vowel \textipa{/\textrhookschwa/}, % /ɚ/
which can be heard in the words like \textit{b\textbf{ir}d} \textipa{[\textprimstress b\textrhookschwa \textlengthmark d]} % [ˈbɚːd]
and \textit{wat\textbf{er}} \textipa{[\textprimstress wA\textlengthmark R\textrhookschwa]}. % [ˈwɑːɾɚ]
This so-called rhotic vowel is extremely rare among the world's languages, but it appears in Simlish in the words like \textit{bl\textbf{ur}sh} \textipa{[\textprimstress bl\textrhookschwa \textlengthmark \textesh]} % [ˈblɚːʃ]
-- 'excuse me' or \textit{litz\textbf{er}gam} \textipa{[\textprimstress lItz\textrhookschwa g\ae m]} % [ˈlɪtzɚgæm]
-- 'thank you' \cite{kirce2024language}.

The strong influence of American~English is not surprising because the voice actors creating the language are Americans.
Even though they speak gibberish, they are still using the sounds of their native language because those are the sounds that they are the most familiar with. 
It is worth noting that the Simlish in the first game sounds less ``English-like'' than the Simlish in the later games. This is an artificial example of the influence that language contact can have on sound shifts over time. Therefore, the Simlish phonology has taken on more features of English as a result of extended contact with the English-speaking world \cite{ejm2020phonology}.

Although the influence of American~English is obvious, there are some differences between the two. For example, Simlish does not have dental phonemes, particularly \textipa{/T/} % /θ/
and \textipa{/D/} % /δ/
(sounds at the beginning of words \textit{\textbf{th}ink} and \textit{\textbf{th}e} respectively), which are very common in many dialects of English. Another difference is consonant clusters. English allows up to three consonants at the beginning of a word (e.g., \textit{\textbf{str}ong}) and four or five consonants at the end of a word (e.g., \textit{si\textbf{xths}} or \textit{a\textbf{ngsts}}). Simlish allows a maximum of two consonants. However, these consonants may be used in unorthodox ways in comparison to English (e.g., \textit{bwu} for 'blue') \cite{ejm2020phonology}.

Many English speakers compare Simlish to babbling or baby talk. After examining the phonology, this does make sense. Simlish allows a lot of consonant~+~\textipa{/w/} % /w/
clusters that are typical for children's early attempts to say words containing a consonant~+~\textipa{/l/} % /l/
or consonant~+~\textipa{/r/}, % /r/
and do not exist in adult English \cite{ejm2020phonology}.

\section*{Writing system}

In The~Sims~1\footnote{The first game ``The~Sims'' is often referred to as ``The~Sims~1'' for clarity.}, any kind of text is usually avoided in favor of pictograms or dingbats. For example, the stop sign in The~Sims is a red octagon with a flat, white hand \cite{atwood2007software}. But by The~Sims~2, we see a Simlish-specific writing system appear on signs and television \cite{kirce2024language}. However, there is no official Simlish alphabet or set of glyphs used. Throughout the series, multiple different custom fonts have been used to represent texts in Simlish, making it unreadable. I assume that this further supports the freedom of interpretation that Simlish offers.

As Simlish appears more and more in the newer games, some interesting variations have emerged.
In The~Sims~2:~University, we can see Simlish variations of Greek letters.
But a particularly interesting variation was introduced with The~Sims~4:~Snowy~Escape and is called ``Simji''. It mixes and morphs Hiragana, Katakana and Kanji, the three writing systems used in Japanese orthography, into a sort of ``Japanese version of Simlish''. The name Simji therefore comes from the fusion of the words Simlish and Kanji \cite{kirce2024language}. 

\section*{Linguistic patterns}

As a result of people using a language over an extended period of time, some patterns will emerge. One of these that we can frequently find in Simlish is called reduplication. Reduplication is a morphological process in which a word or part of a word is repeated exactly or with a slight change. It is a very common linguistic feature across world's languages and serves various grammatical and semantic functions. For example, in Malay, it can be used to form non-exhaustive plurals like \textit{burung-burung} meaning 'all those birds'. In English, an example of full reduplication, which means repeating the whole word, is \textit{bye-bye}. Examples of partial reduplication with only changing a vowel could be \textit{chit-chat}, \textit{flip-flop} or \textit{knick-knack} \cite{kirce2024language}.

In Simlish, we see a lot of examples of full reduplication, like \textit{sul sul} -- 'hello', \textit{dag dag} -- 'goodbye', \textit{choo waga choo choo} -- 'something is in the way', \textit{baba} -- 'I'm pregnant', \textit{renato renato} -- 'go away' and there are many more.
We could even argue that there is some partial reduplication in words like \textit{nooboo} -- 'baby' or \textit{hooba noobie} -- 'what's up'. But since Simlish has no known grammar, we cannot really say how reduplication works in Simlish or what function it serves \cite{kirce2024language}.

\section*{Semantics and English translations}

Despite Simlish starting as an improvised language, some of the words have been used so often in certain situations that they developed a meaning. Although there is no official dictionary of Simlish, fans have been creating unofficial dictionaries \cite{simlish2012simlish, beck2023simlish}, assigning meanings to the gibberish words based on the context in which sims say them in and the emotions they express.
Now, some of the words have official translations confirmed by EA\footnote{EA (Electronic~Arts) is the company behind The~Sims series.} -- some of the most well known being \textit{sul sul} meaning 'hello', \textit{dag dag} meaning 'goodbye', \textit{nooboo} meaning 'baby' or \textit{chumcha} meaning 'pizza' \cite{ea2004thesims2.com}.

Nevertheless, the voice acting for Simlish is still largely improvised. This is confirmed in a more recent interview with Maxis\footnote{Maxis is the studio developing The Sims, a subsidiary of EA.} voice director, who says that the regular cast of voice actors are improvising 90\% of the time \cite{bassi2023how}. But they also mention that translating existing popular songs into Simlish is a different process.

\section*{Famous songs in Simlish}

Something very interesting and unique about Simlish is that many popular songs have been translated into it. Artists like Katy~Perry, Black~Eyed~Peas, Lily~Allen, Depeche~Mode, Jason~Derulo, Anitta and many others recorded their original songs in Simlish \cite{wiki2025songs}, bringing the familiar tunes with a new twist into The~Sims games. As a byproduct, these songs are actually the largest source of Simlish, essentially serving as a ``parallel corpus'' between English and Simlish.

Apparently, EA has a dictionary of Simlish that they use to translate these songs. But wherever it falls short, they make up new words, all in a way that fits the melody and rhythm of the original song \cite{bassi2023how}. Unfortunately, the dictionary is not public and they probably will not share it. However, fans have been transcribing the Simlish lyrics and thus, kind of reconstructing the translations. Obviously, most of the fans are not linguists, and therefore they do not use phonetic transcription, but rather the English alphabet. This means the transcriptions are inexact and inconsistent. For example, the translation of the word 'you' is often transcribed as \textit{voo} but sometimes as \textit{vou} or even \textit{vous}.  

Because the language is improvised and has been developing over the years, the translations of the songs are not always consistent either. For instance, the word 'baby', both in the sense of a child and as a term of endearment, is confirmed to be \textit{nooboo}. But as \cite{kirce2024language} pointed out in her video, in Bryan Rice's song ``There For You'', 'baby' is translated as \textit{bwayzay}. This is just one of many examples of inconsistencies in the Simlish translations, although it is fair to say that the translations are becoming more consistent in the newer songs.

Examples of English and Simlish lyrics side by side can be found in the paper by \cite{brouwer2022“lass} or in the videos on \cite{grengoddess2025grengoddess} YouTube channel.

\section*{Comparison with other fictional languages}

Simlish is truly unique even in the realm of fictional languages. Due to its improvised nature, it can hardly compete with other constructed languages such as Star~Trek's Klingon or J. R. R. Tolkien's Elvish languages Quenya and Sindarin. These are full-fledged languages with grammar, linguistic rules and vocabulary, designed by linguists for real communication. But that was never the goal of Simlish. The different intention is nicely illustrated when we compare Simlish to another constructed language ``Dovahzul'', the dragon language created for the video game Skyrim\footnote{The Elder Scrolls V: Skyrim is the full title of the game.}.

Dovahzul is also a fully developed, functional language with its own grammar, rules, alphabet and vocabulary. Its purpose is to further enhance the immersion and storytelling in the magical world of Skyrim. In contrast, Simlish adds a playful and humorous element to The~Sims, breaking down the language barriers and allowing all players to understand and feel the emotions of sims through this seemingly unintelligible gibberish \cite{bracchi2023invented}.

\section*{Conclusion}

All in all, Simlish is a unique and fascinating language that has become an essential part of The~Sims series. It is a language that is not meant to be understood in the traditional sense, but rather to convey emotions and the meaning of the words through its sounds. It was never intended to be a full-fledged language with grammar and structure. However, during its now 20+ years of existence, Simlish has been developing and expanding. Players and fans have been creating dictionaries, transcribing Simlish songs and even teaching how to speak Simlish. Although it is still a gibberish language, some very simple conversations can be held in Simlish, and it will be interesting to see how the language will continue to evolve in the future.


%\section{Introduction}
%
%These instructions are for authors submitting papers to *ACL conferences using \LaTeX. They are not self-contained. All authors must follow the general instructions for *ACL proceedings,\footnote{\url{http://acl-org.github.io/ACLPUB/formatting.html}} and this document contains additional instructions for the \LaTeX{} style files.
%
%The templates include the \LaTeX{} source of this document (\texttt{acl\_latex.tex}),
%the \LaTeX{} style file used to format it (\texttt{acl.sty}),
%an ACL bibliography style (\texttt{acl\_natbib.bst}),
%an example bibliography (\texttt{custom.bib}),
%and the bibliography for the ACL Anthology (\texttt{anthology.bib}).

%\section{Engines}
%
%To produce a PDF file, pdf\LaTeX{} is strongly recommended (over original \LaTeX{} plus dvips+ps2pdf or dvipdf).
%The style file \texttt{acl.sty} can also be used with
%lua\LaTeX{} and
%Xe\LaTeX{}, which are especially suitable for text in non-Latin scripts.
%The file \texttt{acl\_lualatex.tex} in this repository provides
%an example of how to use \texttt{acl.sty} with either
%lua\LaTeX{} or
%Xe\LaTeX{}.

%\section{Preamble}
%
%The first line of the file must be
%\begin{quote}
%\begin{verbatim}
%\documentclass[11pt]{article}
%\end{verbatim}
%\end{quote}
%
%To load the style file in the review version:
%\begin{quote}
%\begin{verbatim}
%\usepackage[review]{acl}
%\end{verbatim}
%\end{quote}
%For the final version, omit the \verb|review| option:
%\begin{quote}
%\begin{verbatim}
%\usepackage{acl}
%\end{verbatim}
%\end{quote}
%
%To use Times Roman, put the following in the preamble:
%\begin{quote}
%\begin{verbatim}
%\usepackage{times}
%\end{verbatim}
%\end{quote}
%(Alternatives like txfonts or newtx are also acceptable.)
%
%Please see the \LaTeX{} source of this document for comments on other packages that may be useful.
%
%Set the title and author using \verb|\title| and \verb|\author|. Within the author list, format multiple authors using \verb|\and| and \verb|\And| and \verb|\AND|; please see the \LaTeX{} source for examples.
%
%By default, the box containing the title and author names is set to the minimum of 5 cm. If you need more space, include the following in the preamble:
%\begin{quote}
%\begin{verbatim}
%\setlength\titlebox{<dim>}
%\end{verbatim}
%\end{quote}
%where \verb|<dim>| is replaced with a length. Do not set this length smaller than 5 cm.

%\section{Document Body}
%
%\subsection{Footnotes}
%
%Footnotes are inserted with the \verb|\footnote| command.\footnote{This is a footnote.}
%
%\subsection{Tables and figures}
%
%See Table~\ref{tab:accents} for an example of a table and its caption.
%\textbf{Do not override the default caption sizes.}
%
%\begin{table}
%  \centering
%  \begin{tabular}{lc}
%    \hline
%    \textbf{Command} & \textbf{Output} \\
%    \hline
%    \verb|{\"a}|     & {\"a}           \\
%    \verb|{\^e}|     & {\^e}           \\
%    \verb|{\`i}|     & {\`i}           \\
%    \verb|{\.I}|     & {\.I}           \\
%    \verb|{\o}|      & {\o}            \\
%    \verb|{\'u}|     & {\'u}           \\
%    \verb|{\aa}|     & {\aa}           \\\hline
%  \end{tabular}
%  \begin{tabular}{lc}
%    \hline
%    \textbf{Command} & \textbf{Output} \\
%    \hline
%    \verb|{\c c}|    & {\c c}          \\
%    \verb|{\u g}|    & {\u g}          \\
%    \verb|{\l}|      & {\l}            \\
%    \verb|{\~n}|     & {\~n}           \\
%    \verb|{\H o}|    & {\H o}          \\
%    \verb|{\v r}|    & {\v r}          \\
%    \verb|{\ss}|     & {\ss}           \\
%    \hline
%  \end{tabular}
%  \caption{Example commands for accented characters, to be used in, \emph{e.g.}, Bib\TeX{} entries.}
%  \label{tab:accents}
%\end{table}
%
%As much as possible, fonts in figures should conform
%to the document fonts. See Figure~\ref{fig:experiments} for an example of a figure and its caption.
%
%Using the \verb|graphicx| package graphics files can be included within figure
%environment at an appropriate point within the text.
%The \verb|graphicx| package supports various optional arguments to control the
%appearance of the figure.
%You must include it explicitly in the \LaTeX{} preamble (after the
%\verb|\documentclass| declaration and before \verb|\begin{document}|) using
%\verb|\usepackage{graphicx}|.
%
%\begin{figure}[t]
%  \includegraphics[width=\columnwidth]{example-image-golden}
%  \caption{A figure with a caption that runs for more than one line.
%    Example image is usually available through the \texttt{mwe} package
%    without even mentioning it in the preamble.}
%  \label{fig:experiments}
%\end{figure}
%
%\begin{figure*}[t]
%  \includegraphics[width=0.48\linewidth]{example-image-a} \hfill
%  \includegraphics[width=0.48\linewidth]{example-image-b}
%  \caption {A minimal working example to demonstrate how to place
%    two images side-by-side.}
%\end{figure*}

%\subsection{Hyperlinks}
%
%Users of older versions of \LaTeX{} may encounter the following error during compilation:
%\begin{quote}
%\verb|\pdfendlink| ended up in different nesting level than \verb|\pdfstartlink|.
%\end{quote}
%This happens when pdf\LaTeX{} is used and a citation splits across a page boundary. The best way to fix this is to upgrade \LaTeX{} to 2018-12-01 or later.

%\subsection{Citations}
%
%\begin{table*}
%  \centering
%  \begin{tabular}{lll}
%    \hline
%    \textbf{Output}           & \textbf{natbib command} & \textbf{ACL only command} \\
%    \hline
%    \citep{Gusfield:97}       & \verb|\citep|           &                           \\
%    \citealp{Gusfield:97}     & \verb|\citealp|         &                           \\
%    \citet{Gusfield:97}       & \verb|\citet|           &                           \\
%    \citeyearpar{Gusfield:97} & \verb|\citeyearpar|     &                           \\
%    \citeposs{Gusfield:97}    &                         & \verb|\citeposs|          \\
%    \hline
%  \end{tabular}
%  \caption{\label{citation-guide}
%    Citation commands supported by the style file.
%    The style is based on the natbib package and supports all natbib citation commands.
%    It also supports commands defined in previous ACL style files for compatibility.
%  }
%\end{table*}
%
%Table~\ref{citation-guide} shows the syntax supported by the style files.
%We encourage you to use the natbib styles.
%You can use the command \verb|\citet| (cite in text) to get ``author (year)'' citations, like this citation to a paper by \citet{Gusfield:97}.
%You can use the command \verb|\citep| (cite in parentheses) to get ``(author, year)'' citations \citep{Gusfield:97}.
%You can use the command \verb|\citealp| (alternative cite without parentheses) to get ``author, year'' citations, which is useful for using citations within parentheses (e.g. \citealp{Gusfield:97}).
%
%A possessive citation can be made with the command \verb|\citeposs|.
%This is not a standard natbib command, so it is generally not compatible
%with other style files.

%\subsection{References}
%
%\nocite{Ando2005,andrew2007scalable,rasooli-tetrault-2015}
%
%The \LaTeX{} and Bib\TeX{} style files provided roughly follow the American Psychological Association format.
%If your own bib file is named \texttt{custom.bib}, then placing the following before any appendices in your \LaTeX{} file will generate the references section for you:
%\begin{quote}
%\begin{verbatim}
%\bibliography{custom}
%\end{verbatim}
%\end{quote}
%
%You can obtain the complete ACL Anthology as a Bib\TeX{} file from \url{https://aclweb.org/anthology/anthology.bib.gz}.
%To include both the Anthology and your own .bib file, use the following instead of the above.
%\begin{quote}
%\begin{verbatim}
%\bibliography{anthology,custom}
%\end{verbatim}
%\end{quote}
%
%Please see Section~\ref{sec:bibtex} for information on preparing Bib\TeX{} files.

%\subsection{Equations}
%
%An example equation is shown below:
%\begin{equation}
%  \label{eq:example}
%  A = \pi r^2
%\end{equation}
%
%Labels for equation numbers, sections, subsections, figures and tables
%are all defined with the \verb|\label{label}| command and cross references
%to them are made with the \verb|\ref{label}| command.
%
%This an example cross-reference to Equation~\ref{eq:example}.

%\subsection{Appendices}
%
%Use \verb|\appendix| before any appendix section to switch the section numbering over to letters. See Appendix~\ref{sec:appendix} for an example.

%\section{Bib\TeX{} Files}
%\label{sec:bibtex}
%
%Unicode cannot be used in Bib\TeX{} entries, and some ways of typing special characters can disrupt Bib\TeX's alphabetization. The recommended way of typing special characters is shown in Table~\ref{tab:accents}.
%
%Please ensure that Bib\TeX{} records contain DOIs or URLs when possible, and for all the ACL materials that you reference.
%Use the \verb|doi| field for DOIs and the \verb|url| field for URLs.
%If a Bib\TeX{} entry has a URL or DOI field, the paper title in the references section will appear as a hyperlink to the paper, using the hyperref \LaTeX{} package.

%\section*{Limitations}
%
%Since December 2023, a "Limitations" section has been required for all papers submitted to ACL Rolling Review (ARR). This section should be placed %at the end of the paper, before the references. The "Limitations" section (along with, optionally, a section for ethical considerations) may be up to one page and will not count toward the final page limit. Note that these files may be used by venues that do not rely on ARR so it is recommended to verify the requirement of a "Limitations" section and other criteria with the venue in question.

%\section*{Acknowledgments}
%
%This document has been adapted
%by Steven Bethard, Ryan Cotterell and Rui Yan
%from the instructions for earlier ACL and NAACL proceedings, including those for
%ACL 2019 by Douwe Kiela and Ivan Vuli\'{c},
%NAACL 2019 by Stephanie Lukin and Alla Roskovskaya,
%ACL 2018 by Shay Cohen, Kevin Gimpel, and Wei Lu,
%NAACL 2018 by Margaret Mitchell and Stephanie Lukin,
%Bib\TeX{} suggestions for (NA)ACL 2017/2018 from Jason Eisner,
%ACL 2017 by Dan Gildea and Min-Yen Kan,
%NAACL 2017 by Margaret Mitchell,
%ACL 2012 by Maggie Li and Michael White,
%ACL 2010 by Jing-Shin Chang and Philipp Koehn,
%ACL 2008 by Johanna D. Moore, Simone Teufel, James Allan, and Sadaoki Furui,
%ACL 2005 by Hwee Tou Ng and Kemal Oflazer,
%ACL 2002 by Eugene Charniak and Dekang Lin,
%and earlier ACL and EACL formats written by several people, including
%John Chen, Henry S. Thompson and Donald Walker.
%Additional elements were taken from the formatting instructions of the \emph{International Joint Conference on Artificial Intelligence} and the \emph{Conference on Computer Vision and Pattern Recognition}.

% Bibliography entries for the entire Anthology, followed by custom entries
%\bibliography{anthology,custom}
% Custom bibliography entries only
\bibliography{custom}

%\appendix
%
%\section{Example Appendix}
%\label{sec:appendix}
%
%This is an appendix.

\end{document}
